%\documentclass[conference]{IEEEtran}
\documentclass[sigconf]{acmart}

\settopmatter{printacmref=false} % Removes citation information below abstract
\renewcommand\footnotetextcopyrightpermission[1]{} % removes footnote with conference information in first column
\pagestyle{plain} % removes running headers

\usepackage{multirow}
\usepackage{dblfloatfix}
\usepackage{tikz}
\usepackage{pgfplots}
\usetikzlibrary{backgrounds, positioning, fit}
\usetikzlibrary{shapes.geometric}
\usepackage{amsmath}
\usetikzlibrary{patterns}
\usetikzlibrary{pgfplots.groupplots}
\usepackage{subfigure}
\usepackage{float}
\usepackage{wrapfig}


\begin{document}
%\pagestyle{plain}

\title{Quantifying Privacy Leakage in Graph Embedding}


%\author{
%    \IEEEauthorblockN{Vasisht Duddu\IEEEauthorrefmark{1}, Antoine Boutet\IEEEauthorrefmark{1},  Virat Shejwalkar\IEEEauthorrefmark{2}}
%    \IEEEauthorblockA{\IEEEauthorrefmark{1}Univ Lyon, INSA Lyon, Inria, CITI}
%    \IEEEauthorblockA{\IEEEauthorrefmark{2}University of Massachusetts Amherst}
%    \IEEEauthorblockA{vduddu@tutamail.com, antoine.boutet@insa-lyon.fr, vshejwalkar@cs.umass.edu}
%}




%\maketitle

\begin{abstract}


Graph embeddings have been proposed to map graph data to low dimensional space for downstream processing such as node classification and link prediction. With the increasing collection of personal data, graph embeddings can be trained on private and sensitive data. In this work, we quantify the privacy leakage in graph embeddings through three inference attacks targeting Graph Neural Networks. We propose a membership inference attack to infer whether a graph node corresponding to individual user's data was member of the model's training or not. We consider a blackbox setting where the adversary exploits the output prediction scores, and a whitebox setting where the adversary has also access to the released node embeddings. This attack provides an accuracy up to 28\% (blackbox) 36\% (whitebox) beyond random guess by exploiting the distinguishable footprint between train and test data records left by the graph embedding. We propose a Graph Reconstruction attack where the adversary aims to reconstruct the target graph given the corresponding graph embeddings. Here, the adversary can reconstruct the graph with more than 80\% of accuracy and link inference between two nodes around 30\% more confidence than a random guess. We then propose an attribute inference attack where the adversary aims to infer a sensitive attribute. We show that graph embeddings are strongly correlated to node attributes from which the adversary can infer sensitive information such as gender and location.









%Graphs are ubiquitous to model the relationship between nodes representing various types of data.
%Consequently, graph embedding algorithms have been proposed to map graph data to low dimensional space for downstream processing such as node classification and link prediction.
%With the increasing collection and processing of personal data, graph embeddings can be trained on potentially private and sensitive graph data.
%In this work, we study and quantify the privacy leakage in graph embeddings by proposing novel inference attacks specific to Graph based deep learning models.
%Given a blackbox Graph Neural Network for node classification, we propose membership inference attack to infer whether a graph node corresponding to individual user's data was member of the model's training or not by exploiting the output prediction scores.
%In this blackbox setting, we show that the attack gives the adversary an inference advantage as high as 56\% beyond random guess.
%We further extend the attack to whitebox setting where adversary has access to the released node embeddings to indicate an adversary advantage as high as 72\%.
%These attacks exploit the fact graph embedding algorithms leave a distinguishable footprint between train and test data records.
%We then propose Graph Reconstruction attack where the adversary aims to reconstruct the target graph given the corresponding graph embeddings.
%In this threat model, we show the adversary can reconstruct the graph with high precision (AUC) of 0.77 (0.65), 0.72 (0.65) and 0.95 (0.94) for Citeseer, Cora and Pubmed respectively.
%A serious implication of reconstruction is link inference, where the adversary can infer the presence of an edge between two nodes in a graph with high accuracy of 93.39\% (Cora), 90.73\% (Citeseer) and 57.28\% (Pubmed) compared to 50\% random guess.
%Finally, we show that graph embeddings are strongly correlated to sensitive node attributes from which the adversary can infer gender and location for LastFM data (F1: 0.65 for DeepWalk and 0.83 for Node2Vec) and Facebook (F1: 0.59 for DeepWalk and 0.61 for Node2Vec).


%We further extend the attack to whitebox setting where adversary has access to the released node embeddings to indicate an adversary advantage as high as 72\%.
%These attacks exploit the fact graph embedding algorithms leave a distinguishable footprint between train and test data records.
%We then propose Graph Reconstruction attack where the adversary aims to reconstruct the target graph given the corresponding graph embeddings.
%In this threat model, we show the adversary can reconstruct the graph with high precision (AUC) of 0.77 (0.65), 0.72 (0.65) and 0.95 (0.94) for Citeseer, Cora and Pubmed respectively.
%A serious implication of reconstruction is link inference, where the adversary can infer the presence of an edge between two nodes in a graph with high accuracy of 93.39\% (Cora), 90.73\% (Citeseer) and 57.28\% (Pubmed) compared to 50\% random guess.
%Finally, we show that graph embeddings are strongly correlated to sensitive node attributes from which the adversary can infer location from LastFM data (F1: 0.65 for DeepWalk and 0.83 for Node2Vec) and gender from Facebook data (F1: 0.59 for DeepWalk and 0.61 for Node2Vec).

\end{abstract}
\keywords{Privacy Leakage, Inference Attacks, Graph Neural Networks, Graph Embeddings.}
%\begin{IEEEkeywords}
%Privacy Leakage, Inference Attacks, Graph Neural Networks, Graph Embeddings.
%\end{IEEEkeywords}

\maketitle

\section{Introduction}\label{introduction}

Large scale real-world systems are typically modelled in the form of graphs: online social networks, world wide web, citation networks and biomedical datasets, which represent the entities as nodes and their relationship with edges~\cite{zhou2018graph}.
Traditional Deep Neural Networks fail to capture the nuances of structured data but a specific class of algorithms, namely, Graph Neural Networks (GNNs) have shown state of the performance on such complex graph data for node classification, link prediction etc.
Such models requires large data which raises the question of privacy if they are trained with private and potentially sensitive data.
Consider a graph capturing the outbreak of a disease where the nodes represent the individuals, medical symptoms as the node features and the edges indicating the disease transmission.
Typically, in such datasets a GNN provides state of the art performance for predicting disease for an arbitrary user in the graph (node classification) and determining the future outbreak (link prediction).
For such a model, an adversary can infer the health status of a particular user (node in graph) by identifying whether the user was part of the training data or not.
Further, the adversary can potentially reconstruct the graph from the low dimensional embeddings enabling the adversary to extract the sensitive input to the models.
Finally, graph embeddings capture important semantics from the input graph while maintaining the contextual information in the form of preferential connection which can be exploited to infer sensitive attributes about an individual.
These three privacy attacks, namely, node inference, graph reconstruction and attribute inference, are examples of a direct privacy violation of the individual which can further be used without user consent. % for deciding medical insurance premium or job hiring.
Further, companies spend enormous resources to annotate the training dataset to achieve state of the art performance and such attacks inferring training data also violates the Intellectual Property.
Hence, studying and mitigating membership inference risk is crucial specifically with the onset of data protection laws such as HIPAA\footnote{https://www.hhs.gov/hipaa/index.html} and GDPR\footnote{https://gdpr.eu/}.

In the context of Machine Learning, a privacy violation occurs when an adversary infers something about a \textit{particular} user's data record in the training dataset which cannot be inferred from other models trained on similar data distribution~\cite{7958568,8835245}.
This information leakage is quantified using the success of inference attacks: membership inference and attribute inference and reconstruction attacks.
In attribute inference attacks, the attacker infers sensitive features of an individual's data record used in model's training.
A stronger case of attribute inference is where the attacker reconstructs a portion of the sensitive training data itself, i.e, data reconstruction attack.
In case of node membership inference, the adversary traces a particular individual's record to the training dataset, i.e., identify whether a given data record was a member of the training data.
Prior literature in privacy attacks focus on models trained on non-graph data including text, images and speech to study the vulnerability to membership inference~\cite{7958568}, attribute inference~\cite{10.1504/IJSN.2015.071829}, property inference~\cite{10.1145/3243734.3243834}, model inversion~\cite{10.1145/2810103.2813677} attacks as well as model parameter and hyperparameter stealing attacks~\cite{10.5555/3241094.3241142,8418595}.
However, the privacy risk in Graph-based machine learning models under adversarial setting is not yet explored.
In this work, we evaluate the vulnerability of graph embedding algorithms along with GNNs against the threat from three privacy attacks: node membership inference,

\noindent\textbf{Contributions.}

\noindent\textbf{Paper Organization.}  We introduce the basic idea behind graph embedding algorithms and GNNs in Section~\ref{background}, followed by a detailed attack taxonomy and threat models for the three proposed privacy attacks: Node Membership Inference, Graph Reconstruction and Node Attribute Inference attacks in Section~\ref{attack}.
We describe the experimental setup with the baselines and dataset descriptions in Section~\ref{setup}.
Finally, the evaluations of the proposed attacks under different threat models and settings are given in Section~\ref{evaluation}.

\section{Background}\label{background}


\subsection{Graph Embedding}


Different graph embedding algorithms to embed both the entire graphs as well as the nodes have been well studied~\cite{node2vec,deepwalk,line,sdme,graph2vec,harp}

adversarial attacks on graph embeddings have been explored~\cite{nodepoison} but the privacy risks of releasing embedding on the user's sensitive data are not well understood and studied

graph embedding are important~\cite{tutorial}


Privacy risks in machine learning have been explored using membership inference attacks~\cite{membershipinf}, property inference~\cite{propertyinf} and attribute inference attacks~\cite{attributeinf,attributeinf2,overlearninginf}.
Further, model extraction attacks aim to reconstruct the target model architecture via side channels~\cite{csinn,timing} and steal the functionality by training the reconstructed architecture on target model predictions as labels~\cite{stealml}.

attack model~\cite{gae,vgae}

connectivity Information in graphs are represented as adjacency matrix which is then used for various applications such as attribute prediction, clustering, link prediction, node classification.
Some algorithms transform these adjacency matrix (connectivity information) and convert it to a low dimension latent representation which can be used as features.




\subsection{Graph Neural Networks}

A large number of real-world applications require processing graph data which contains rich relational information between different entities (e.g., online social media, disease outbreaks, recommendation engines, knowledge graphs and navigation systems).
%Machine Learning provides solutions to extract knowledge from such data, but transforming the data and models is not trivial \cite{zhou2018graph}.
Deep Learning and more precisely Convolutional Neural Networks have shown tremendous performance over non-graph data such as images by capturing the spatial relation between pixels of image and extracting features over multiple layers.
However, this machine learning scheme has shown its limits for graph data and the learning on such data is still challenging~\cite{zhou2018graph}.
Indeed, the models have to capture the connections in the data while ensuring invariance of graph data representation, even without fixed ordering between the nodes (i.e., the adjacency matrix representing the connections between nodes varies but still results in the same graph). %different

To overcome this limitation, Graph Neural Networks (GNNs) have been introduced.
GNNs transform models operating on low dimensional euclidean datasets (i.e., such as images) to graph data by mapping these graph data into a low dimensional feature embedding space.
Specifically, the parameters of the embedding function are updated to improve the feature representation of the graph nodes while maintaining the original properties.

%\noindent\textbf{Training GNNs.}
Consider a graph $G=(V,E)$ where $V$ represents the vertex set consisting of nodes \{$v_1,...,v_n$\} where the connections between the edges $E$ is represented as a symmetric, sparse adjacency matrix $A$ $\in$ $R_{nxn}$ where $a_{ij}$ denotes the edge weight between nodes with $a_{ij}= 0$ for missing edges.
The graph data is pre-processed to obtain features for each node and node connections as matrices to get the training data $D_{train}$.
The training of GNNs relies on message passing algorithm which is the weighted aggregation of features of neighbouring nodes $\mathcal{N}(v)$ to compute the feature of a particular node $v$.
The loss over the resultant classification for the node $v$ is then backpropagated to update the model weights for aggregation.
Given the features $x$ of a single node, the GNN produces an output label $f(x;W)$ which captures the probability of the input node with features $x$ belonging to a particular class.

Consider a $N\times D_F$ feature matrix $X$ where N is the number of nodes, $D_F$ is the number of node features and an adjacency matrix $A$ which captures the representation of graph structure in matrix form.
The output of a layer with $F$ features takes the feature matrix along with the adjacency matrix as input to produce a $N\times F$ matrix as an output.
\begin{equation}
H^{(l+1)} = f(H^{(l)}, A)
\end{equation}
with $H(0)=X$ and $H(L)=Z$, $L$ being the number of layers and $H$ is the intermediate activation.
The parameterized embedding function $f$ learns to map the node features to a lower dimensional embedding space.
In this work, we evaluate the privacy leakage on four state of the art GNNs which have different embedding functions $f$:
%In this work, we evaluate the privacy leakage on four state of the art GNNs: Graph Convolutional Networks (GCN), GraphSAGE, Graph Attention Networks (GAT) and Topology Adaptive Graph Convolutional Networks (TAGCN), which have different embedding functions $f$.

Graph Convolutional Network (GCN) \cite{Kipf2016tc}. In the simple model of GNNs mentioned above, we aggregate the features of neighbouring nodes but not the node itself.
In order to compensate this, GCN adds the identity matrix $I$ to the original adjacency matrix $\hat{A} = A + I$.
Further, the adjacency matrix is not normalized and the multiplication on $A$ will change the scale of the node features.
GCN addresses this by normalizing $A$ as $D^{-1}A$ where $D$ is the diagonal node degree matrix and results in averaging of neighbouring node features.
An additional trick is to use a symmetric normalization as $D^{-\frac{1}{2}}\hat{A}D^{-\frac{1}{2}}$. This results in the following propagation rule for GCN:

\begin{equation}
f(H^{(l)}, A) = \sigma\left( \hat{D}^{-\frac{1}{2}}\hat{A}\hat{D}^{-\frac{1}{2}}H^{(l)}W^{(l)}\right)
\end{equation}
with $\hat{D}$ being the diagonal node degree matrix of $\hat{A}$.

GraphSAGE \cite{NIPS20176703}.
Graph Attention Networks (GAT) \cite{velickovic2018graph}.
Topology Adaptive GCN (TAGCN) \cite{du2018topology}.

\section{Attack Taxonomy}\label{attack}


Other than membership inference inference attacks in Graph NNs as described in [], we propose four novel attack surfaces for Graph models based on the leakage from graph embeddings.

\begin{figure}[!htb]
\centering
\includegraphics[width=0.85\linewidth]{./figures/Attacks/MIA.pdf}
\caption{Example of a parametric}
\end{figure}



\begin{figure}
    \centering
    \begin{minipage}[b]{1\linewidth}
    \centering

    \subfigure[Output Distribution for all records]{
   	\label{fig:mem_soft_label}
    \includegraphics[width=\linewidth]{./figures/Attacks/reconstruction.pdf}
    }

    \subfigure[Output Distribution for all records]{
    \label{fig:mem_soft_label}
    \includegraphics[width=0.6\linewidth]{./figures/Attacks/reconstruction2.pdf}
    }
    \end{minipage}
    \caption{Distribution of the confidence score vectors of the target classifier on the training data and test data of class 29 in the Purchase100 dataset. Each color represents one data record.}
    \label{fig:soft_label}
\end{figure}



Threat Model: blackbox..

Adversary prior Knowledge

\subsection{Graph Reconstruction Attack}

Adversary Goal

Attack Methodology

\subsection{Stealing Model Functionality}


\subsection{Link Inference Attack}

Link Inference attacks is a binary classification problem where the adversary aims to infer whether there exists a links between two nodes in the graph.
This translates to identifying whether two people know each other in case of online social networks and identifying the friendship circle which can violate the privacy of the individual.
Link Inference attacks naturally follow from the reconstruction attack where given the reconstructed graph, the adversary can check for an edge between two users using the adjacency matrix.
Write link inference as checking the adjacency matrix...


\subsection{Attribute Inference Attack}

Given an embedding of the graph node $\Psi$

Adversary Goal

Attack Methodology



\subsection{Embedding MIA}

The adversary in a whitebox setting has access to the model output predictions $f(x; W)$ for an input $x$ as well as the model parameters $W$.
This allows the adversary to compute the intermediate computations after each layer.
This is a strong adversary assumption but practical in cases such as federated learning where the intermediate computations and parameters can be observed~\cite{8835245,DBLP:conf/sp/MelisSCS19}.

%\noindent\textbf{Attack Motivation.}
As explained Section~\ref{graphnn}, GNNs compute the low dimensional feature embedding for the input graph data.
The parameters of the embedding function are updated in each iteration of training and tuned specifically for high performance on the train data resulting in a distinguishable footprint between feature embedding of train and test data points.
Figure~\ref{embedding} illustrates this rationale by plotting feature embedding of train and test records for the three datasets after a dimension reduction using 2D-TSNE algorithm~\cite{vanDerMaaten2008}.



%\noindent\textbf{Attack Methodology.}
The attack is unsupervised since we assume the adversary has no prior knowledge to map the intermediate feature embeddings to a membership value. % (available as auxiliary knowledge in shadow model).
The adversary trains an encoder-decoder network in unsupervised fashion to map the intermediate embedding to a single membership value.
For an input $x$, encoder $f_{enc}()$ generates a scalar membership value which is passed to decoder $f_{dec}(f_{enc}(x))$ to obtain $x$ by minmizing reconstruction loss: $||x - f_{dec}(f_{enc}(x))||_2^2$.
Given the membership values for different training and testing data points, K-Means clustering is used to cluster the nodes into two classes (members and non-members).
For any new input, the adversary can then use this clustering to map it as members or non-members of the training data.
This novel whitebox attack exploits the difference in embedding representation between members and non-members of training data which is not possible for Deep Neural Networks trained on euclidean data where the intermediate activations are abstract (generalize well) and cannot be used to distinguish members and non-members~\cite{8835245}.

\section{Experiment Setup}\label{setup}

In this section, we present the considered datasets, embedding algorithms, evaluation metrics and the methodology.

\subsection{Datasets}

For the membership inference and graph reconstruction attack, we consider three standard benchmarking datasets: Pubmed, Citeseer and Cora.
For the attribute inference attack, in turn, we consider two social networking datasets with anonymized sensitive attributes: Facebook\footnote{http://snap.stanford.edu/data/ego-Facebook.html} and LastFM\footnote{http://snap.stanford.edu/data/feather-lastfm-social.html}.

\noindent\textbf{Pubmed.} The Pubmed Diabetes dataset consists of 19,717 scientific publications from PubMed database pertaining to diabetes classified into one of three classes. The citation network consists of 44,338 links. Each publication in the dataset is described by a TF/IDF weighted word vector from a dictionary which consists of 500 unique words.
We use 60 train samples, 500 validation samples and 1,000 test samples.

\noindent\textbf{Citeseer.} The CiteSeer dataset consists of 3,312 scientific publications classified into one of six classes.
The citation network consists of 4,732 links.
Each publication is described by a 0/1-valued word vector indicating the absence/presence of the corresponding word from the dictionary.
%Each publication in the dataset is described by a 0/1-valued word vector indicating the absence/presence of the corresponding word from the dictionary.
The dictionary consists of 3,703 unique words.
The number of training samples is 120, 500 validation samples and 1,000 test samples.

\noindent\textbf{Cora.} The Cora dataset consists of 2,708 scientific publications classified into one of seven classes.
The citation network consists of 5,429 links. Each publication is described by a 0/1-valued word vector indicating the absence/presence of the corresponding word from the dictionary.
The dictionary consists of 1,433 unique words.
For training 140 samples are used, 300 validation samples and 1,000 test samples.

\noindent\textbf{Facebook.} The dataset comprises of 4,039 nodes representing different user accounts on the social network connected with each other through 88,234 edges.
Each user node has different features including the gender, education, hometown etc.
%Each user node comprises of different features including the gender, education, hometown etc.
The user information has been anonymized through pseudonymization and the interpretation of the features have been obscured (i.e, attributes 'Male' and 'Female' have been replace with 'Gender 1' and 'Gender 2', respectively).


\noindent\textbf{LastFM.} The dataset was collected in March 2,020 using the public API provided by the social network specifically created for users from Asian countries.
The dataset has 7,624 nodes connected together with 27,806 edges based on mutual follower relationships.
Each user has attributes such as the music and artists they likes, location etc.


\subsection{Embedding Algorithms}


\noindent For the purpose of our experiments, we consider two classes of embedding algorithms: GNNs and random walk based.
We consider the following GNN based embedding techniques:

\noindent\textbf{Graph Convolutional Network (GCN)~\cite{Kipf2016tc}.} GCN computes the target node features from neighbouring nodes using matrix factorization, by normalizing adjacency matrix $A$ as $D^{-1}A$ where $D$ is the diagonal node degree matrix and results in averaging of neighbouring node features.
An additional trick is to use a symmetric normalization as $D^{-\frac{1}{2}}\hat{A}D^{-\frac{1}{2}}$.

\noindent\textbf{GraphSAGE~\cite{NIPS20176703}.} GraphSAGE extends the operations in GCN to more generic functions for transformation and aggregating of node features.
While the embedding of graph data in GCN relies on matrix factorization, GraphSAGE uses node feature aggregation using mean, LSTM and pooling to learn the embedding function.

\noindent\textbf{Graph Attention Networks (GAT)~\cite{velickovic2018graph}.} %There are weights associated with features during aggregation which are explicitly defined and learnt during training.
Weights associated with features during aggregation are explicitly defined and learnt during training.
GAT implicitly defines the weights using self-attention mechanism over the node features.

\noindent\textbf{Topology Adaptive GCN (TAGCN)~\cite{du2018topology}.} Instead of using the spectral convolutions for learning non-linear graph data, TAGCN proposes to use general K-localized filter convolution in the vertex domain.
It replaces the fixed square filters in traditional spectral GCNs for the grid-structured input data volumes.



\noindent For membership inference, we consider the embeddings from all the above four architectures for the whitebox setting while for the blackbox setting we consider only GraphSAGE algorithm as inductive training graphs models is challenging and GraphSAGE architecture is specifically designed to work in such training settings~\cite{NIPS20176703}.
In case of graph reconstruction attacks, we consider the generic GCN model as the encoder for the attack model.
In case of attribute inference attacks, we consider two state of the art unsupervised graph embedding algorithms based on random walk, namely, Node2Vec~\cite{node2vec} and DeepWalk~\cite{deepwalk}.

\noindent\textbf{DeepWalk~\cite{deepwalk}.} The algorithm creates a transition matrix from the graph and samples random walks from the matrix.
The nodes are viewed as words and the random walks are viewed as sentences and the resulting sequences are passed to Word2Vec and SkipGram~\cite{wordemb} to obtain node embeddings.

\noindent\textbf{Node2Vec~\cite{node2vec}.} This is an extension of DeepWalk which combines Breadth First and Depth First search explorations on the graph to create biased random walks.
The embeddings are computed using Word2Vec algorithm as mentioned above.







\subsection{Metrics}

%\noindent The performance of inference attacks, namely, node membership inference and link inference attacks, the inference accuracy is used to estimate the attack success.

\noindent To estimate the attack success of both membership and link inference, we consider the inference accuracy.

\noindent\textbf{Inference Accuracy.} Membership and link inference are binary classification problems: node is part of the training data or not (membership inference) and there exists a link between any two nodes or not (Link Inference).
Hence, the accuracy of random guess is 50\% and any higher accuracy indicates a privacy leakage about the model's sensitive training data.
In order to compute the additional benefit the adversary gets in terms of performing the attack over random guess, we name 'adversary advantage' a metric computed as: $I_{adv} = 2*(I_{acc}-0.5)$.
%This metric gives an estimate of the information leakage from the model compared to a random guess.
This metric estimates the information leakage from the model compared to a random guess.


%\noindent\textbf{Adversary Advantage.} In order to compute the additional benefit the adversary gets in terms of performing the attack over random guess, we use the adversary advantage as a metric computed as: $I_{adv} = 2*(I_{acc}-0.5)$.
%This gives an estimate of the information leakage from the model compared to a random guess.
%This gives an estimate of the information leakage from the model over and above a completely privacy preserving model (random guess).

\noindent For evaluating the performance of graph reconstruction attacks, we use two main metrics: precision and roc score.

\noindent\textbf{Precision.} The ratio of true positives is given by the precision and estimates the percentage of the predicted samples that are actually in the target graph.
%The ratio of true positives to the sum of true and false positives is given by precision and estimates the percentage of the predicted samples are actually in the target graph data.

\noindent\textbf{ROC-AUC Score.} The ROC curve plots the true positive rate on the y-axis and the false positive rate on the x-axis. The AUC score computes the area under the ROC curve to get how good the model distinguishes between different classes.
For a binary classification problem of graph reconstruction to obtain the binary adjacency matrix, the random guess accuracy is 50\% and any higher accuracy indicates the adversary's advantage in reconstructing the target graph. %data.


\noindent In case of attribute inference attack, we evaluate using the F1 score to balance both the recall and precision.

\noindent\textbf{F1-Score.} This metric computes the harmonic mean between the precision and recall which estimates the percentage of samples in the target graph which are predicted as such.

\subsection{Methodology}

In this work, we specifically focus on inductive training of GNN where the model does not see test nodes during training unlike transductive learning where the entire graph and features are available apriori.
Given the full graph $G_{full}$, we sample a subgraph $G_{train}$ which is used for training the models and evaluate the model performance on the held out graph $G_{full}-G_{train}$.
Such an inductive setting enables the adversary to learn new information about the target model's training graph resulting in a privacy leakage.

\section{Evaluation}\label{evaluation}

In this section, we evaluate the privacy leakage from different attacks.
The code for all the experiments is made publicly available for easy reproducibility\footnote{Anonymized for Submission}.

\subsection{Node Inference Attack from Output Predictions}


The overfitting for GraphSage architecture trained on the three datasets is given in Figure~\ref{fig:NIA}(a).
We evaluate the two blackbox inference attacks: shadow inference and confidence inference, which exploit the output predictions from the models.
Under the confidence inference attacks, the inference accuracy for Cora is 78.28\% with an adversary's advantage of 27.48\%, 63.75\% inference accuracy for Citesser with an adversary's advantage of 56.56\% and 60.89\% inference accuracy for Pubmed datasets with an advantage of 21.78\%.
In case of shadow model attacks, the inference accuracy for Cora is 62.06\% with an advantage of 21.74\%, inference accuracy of 60.87\% with an advantage of 24.12\% for Citeseer and inference accuracy of 55.51\% with 11.02\% advantage for Pubmed.
The node membership leakage is higher in confidence attack compared to shadow model attack.


\begin{figure}[!htb]
    \centering
    \begin{minipage}[b]{1\linewidth}
    \centering
    \subfigure[Generalization Error \hspace{0.5in} (b) Inference Advantage]{
    \label{fig:nonmem_soft_label}
    \includegraphics[width=0.5\linewidth,height=3.5cm, keepaspectratio]{figures/BBMIA/GenErr.pdf}
    \includegraphics[width=0.5\linewidth,height=3.5cm, keepaspectratio]{figures/BBMIA/NodeIA.pdf}
    }
    \end{minipage}
    \caption{Blackbox Node Inference attack uses the output predictions to give adversary an inference advantage.}
    \label{fig:NIA}
\end{figure}



\textbf{Impact of Increasing Number of Layers.} We evaluate the performance of confidence attack on increasing the range of neighbourhood nodes used for aggregating the features (Figure~\ref{fig:numlayers}).
To do that, we extend the range of the message passing algorithm by increasing the number of layers in the GNNs~\cite{klicpera2018combining,Li2018DeeperII}.
On increasing the number of layers from 2 to 6, the inference accuracy decreases by 8\% for GCN, GraphSAGE.
Interestingly, the generalization error increases (train accuracy remains the same but the test accuracy decreases) for Cora, Citeseer and Pubmed, but the inference accuracy continues to decrease which indicates that the influence of preferential connections between different nodes in the graph plays a significant role in influencing the inference accuracy. % than the generalization error.
For large number of layers ($>$ 8 layers) in the GNN, for all the datasets and architectures, the model completely loses it's predictive power.
In general, the inference accuracy as well as prediction accuracy decreases with increasing the range of the message passing algorithm by increasing the layers from 2 to 16.
This implies that the membership privacy leakage is influenced by the structured graph data with preferential connections between different nodes.
Specifically, aggregating features from larger number of nodes results in higher averaging which reduces the distinguishability (over-smoothening of features) as model converges to random walk’s limit distribution~\cite{klicpera2018combining,Li2018DeeperII} which is crucial for inference attacks~\cite{7958568,ndss19salem}.


\begin{figure}[!htb]
\centering
\includegraphics[width=0.4\textwidth,height=3.8cm]{./figures/BBMIA/gsage_numlayers.pdf}
\caption{The inference accuracy and predictive power decreases on increasing the number of layers due to feature oversmoothening from nodes deeper in the graph.}
\label{fig:numlayers}
\end{figure}





\subsection{Node Inference Attack from Graph Embedding}

We exploit the difference in intermediate feature representation of train and test data to perform membership inference attack in a whitebox setting (Table~\ref{whitebox}).
Results show that different models trained on PubMed dataset leak significantly more information between 20\% and 36\% over random guess accuracy.
On the other hand, models trained on Citeseer dataset provide to the adversary an advantage between ~7\% and ~17\% over random guess while for Cora dataset it is between 4\% and 7\%.
The embedding is significantly different for train and test data points for PubMed dataset as seen Figure~\ref{embedding}(a) which result in a higher whitebox membership inference accuracy compared to Cora and Citeseer dataset (Figure~\ref{embedding}(b) and (c)).
The higher accuracy for Pubmed dataset can be attributed to significant distinguishability of features as seen by visually inspecting in Figure~\ref{embedding}.



\begin{figure}[!htb]
\centering
\includegraphics[width=0.3\textwidth,height=5cm]{figures/EmbeddingMIA/whiteboxMIA.pdf}
\caption{Adversary advantage for node membership inference from Graph Embeddings.}
\label{fig:numlayers}
\end{figure}


The success of the unsupervised whitebox attack is attributed to the message passing which updates the parameters (weights) to specifically ensure higher distinguishability between the data points of different classes for high accuracy on training dataset.
Indeed, the parameters are specifically updated to fit the training dataset resulting in a high distinguishability between feature embedding of train and test data records.
Moreover, the feature embedding for the initial layers are useful since for later layers the features are oversmoothened which also reduces accuracy (as seen in increasing the range of nodes of message passing algorithm).




\subsection{Graph Reconstruction Attack}

The success of graph reconstruction is evaluated on the unseen target graph while the model is trained on the train graph.
The test ROC-AUC score for Cora dataset is 0.65 while the average precision is 0.722 while for Citeseer dataset the ROC-AUC score is 0.65 and 0.778 average precision.
In case of Pubmed dataset, .....
The curve for variation of ROC-mAUC score and the average precision for the three datasets on the validation sub graph is given in Figure~\ref{fig:valgraphrecon}.

\begin{figure}[!htb]
    \centering
    \begin{minipage}[b]{1\linewidth}
    \centering
    \subfigure[Validation ROC \hspace{1.2in} (b) Average Precision]{
   	\label{fig:mem_soft_label}
    \includegraphics[width=0.5\linewidth]{figures/Reconstruction/roc.pdf}
    \includegraphics[width=0.5\linewidth]{figures/Reconstruction/ap.pdf}
    }
    \end{minipage}
    \caption{Training curves for ROC-AUC score and Average Precision on the validation graph.}
    \label{fig:valgraphrecon}
\end{figure}

\textbf{Impact of Adversary Knowledge.} On increasing the adversary's knowledge to 50% of the target graph, we observe an increase the attack performance.
Specifically, the ROC-AUC score for Cora increases to 0.76 from 0.65 while the average precision increases to 0.81 from 0.722.
In Citeseer dataset, the AUC-ROC score increases to 0.779 from 0.65 while the average precision increases to 0.828 from 0.778.
Finally, for Pubmed dataset...



\textbf{Link Inference Attack.} A direct implication of graph reconstruction attack is inferring whether there exists a link between two nodes in the network.
This is a binary classification problem.

\begin{wrapfigure}{l}{0.5\linewidth}
  \begin{center}
    \includegraphics[width=\linewidth]{figures/LinkInfer/LinkInfer.pdf}
  \end{center}
  \caption{Link Inference Accuracy Curve over different epochs}
\end{wrapfigure}

For Citeseer dataset, the accuracy of of inference is around 93.39\% while for Cora dataset the inference accuracy is 90.73\%.
The above results are assuming a practical adversary setting with 1585 training data points (30\%) compared to 3166 test data points (60\%) and 527 validation data points (10\%) for Cora dataset.
The same train-test-validation distribution is used for Citeseer dataset with 1366 (30\%) train records, 2731 (60\%) test records and 455 (10\%) validation records.



\subsection{Attribute Inference Attack}

In case of attribute inference attacks, we evaluate two state of the art embedding models: Node2Vec and DeepWalk, using three attack models: Neural Networks (NN), Random Forest (RF) and Support Vector Machine (SVM).
We generate embeddings using the two algorithms on Facebook and LastFM dataset which contain gender and location as sensitive attributes respectively.
That is, the adversary infers user gender as a target sensitive attribute in Facebook dataset classified into one of three classes: Male, Female and Others.
The location target attribute for LastFM dataset is categorized in 18 locations for the users in the network.

\begin{figure}[!htb]
    \centering
    \begin{minipage}[b]{1\linewidth}
    \centering
    \subfigure[LastFM \hspace{1.2in} (b) Facebook]{
   	\label{fig:mem_soft_label}
    \includegraphics[width=0.5\linewidth,height=3.5cm, keepaspectratio]{./figures/AIA/lfm_AIA.pdf}
    \includegraphics[width=0.5\linewidth,height=3.5cm, keepaspectratio]{./figures/AIA/fb_AIA.pdf}
    }

    \end{minipage}
    \caption{F1 score for different attack classifiers to infer sensitive attributes.}
    \label{fig:aia}
\end{figure}

The inference attack performance is given by the F1 score as shown in Figure~\ref{fig:aia}.
For Facebook, the graph embedding using DeepWalk resulted in an F1 score of 0.57 for NN, 0.58 for RF and 0.59 for SVM classifier.
On the other hand, Facebook's Node2Vec embedding showed an F1 score of 0.59, 0.57 and 0.61 respectively for NN, Rf and SVM attack classifier.
In case of LastFM, we found the attack F1 scores for Node2Vec for higher than DeepWalk embeddings.
The F1 score for Node2Vec 0.61, 0.62 and 0.65 corresponding to NN, RF and SVM attack classifier while the F1 score using Node2Vec embeddings are 0.80, 0.83 and 0.83 for NN, RF and SVM.

\textbf{Impact of Adversary's Knowledge.} The performance of the attack model for Facebook dataset did not increase by much.
On increasing the knowledge of the adversary's auxiliary dataset from 30\% to 50\%, the confidence of attack on LastFM dataset increases.
For DeepWalk algorithm, the attack F1 score increases to 0.69 from 0.61 for NN, 0.71 from 0.62 for RF and 0.69 from 0.65 for SVM attack classifier.
On the other hand, for Node2Vec, the attack model F1 score increased to 0.83 from 0.80 for NN, 0.84 from 0.83 for RF and 0.86 from 0.83 for SVM.

%\section{Mitigating Privacy Risks}\label{discuss}

We discuss some potential mitigation strategies to lower the privacy risks from the proposed algorithms and keep the evaluations as future work.

\noindent\textbf{Lower Embedding Precision.} The embeddings are continuous values and capture the semantic information about the graph structure.
These embeddings while reduce the overall dimensionality of the graph data also capture rich information about the input graphs by ensuring that the properties in the graph are still maintained in the embedding.
Lowering the precision of the embedding vector for each node by rounding can help reduce the attack model from learning rich features about the inputs~\cite{membershipinf,nlp}.

\noindent\textbf{Adversarial Examples.} In the proposed attacks, the attacker model typically involves a machine learning algorithm (e.g node and attribute inference, decoder for graph reconstruction).
Adversarial examples are imperceptible noise added to the output prediction to force the target model to misclassify.
Since, the attack models are vulnerable to adversarial examples, the released embedding can be released with an additional adversarial noise to misclassify the target model.
However, the noise added to be optimized to ensure utility for further downstream applications~\cite{attriguard,memguard}.

\noindent\textbf{Adversarial Training.} In case of node membership inference can be modelled as an minimax adversarial training with joint optimization to minimize the model loss using the graph embeddings (e.g GNNs) while maximising the adversary's loss on inferring the sensitive inputs.
Such a joint optimization can be used to update the embeddings such that they maintain the original graph properties while ensuring that the attack model performance is low.
For inference time, such defences have been explored in the context of traditional machine learning to protect against membership inference~\cite{advreg} and preserve privacy of text models~\cite{textembleak}.

\noindent\textbf{Differential Privacy.} One major approach to protect an individual user's privacy is to add noise sampled from Laplacian and Gaussian distribution.
This carefully added noise ensures that the presence or absence of a user's data in the mechanism does not change the model output.
Further, DP provides a theoretical bound on the total privacy leakage from the mechanism (downstream processing from embeddings) on an individual's data point.
Such DP based embeddings have been proposed for both graph and text models~\cite{dptext,dpne}, however, their efficacy against the proposed privacy attacks are yet to be studied as part of the future work.

\section{Related Work}
\label{related}

The wide availability of location and mobility data has been followed by the development of machine learning schemes to predict interests, colocations or other user information~\cite{noulas2009inferring}. 
While traditional approaches define features to characterize users’ mobility useful for prediction tasks, an important pre-processing step for using graph data with machine learning is embedding the high dimensional graph data to a low dimensional representation for easy processing by machine learning algorithms~\cite{yang2019revisiting}.
In this context, GNNs~\cite{zhou2018graph} have shown state of the art performance on such complex graph data for node classification, link prediction etc.
However, the privacy implications of the use of such embeddings have not been fully considered.

Inference attacks that violate data privacy have been explored in the context of traditional machine learning models.
Membership Inference attacks can be deployed in both whitebox~\cite{whitebox} and blackbox~\cite{membershipinf} setting in traditional machine learning algorithms.
These attacks are further extended to collaborative learning~\cite{collabinf,whitebox} and generative models~\cite{logan}.
On the other hand, reconstruction attacks infer private attributes of the inputs passed to the models~\cite{attributeinf, attributeinf2, propertyinf, modelinversion}.
Other privacy attacks aim to extract hyperparameters~\cite{8418595}, reverse engineer the model architecture and parameters using side channels~\cite{timing} or the output predictions~\cite{stealml}.
Memorization of data by Neural Networks has been attributed as a major cause for privacy leakage~\cite{memorize,secretsharer,overlearninginf}.
%
Further, recent works have indicated privacy risks in Graph NNs where an adversary can infer the presence of a link between two nodes using a manual threshold between the distance of two node features~\cite{linksteal}.
This attack however, is subsumed within our more generic attack methodology where we extract the entire adjacency matrix which can be used to infer the presence of links.
Text models have been shown to leak user data through attribute inference and inversion attacks~\cite{textembleak,nlp}. 
However, a direct application of these attacks is not possible in case of high dimensional graphs and requires additional consideration to the network structure making our problem challenging.
Other than privacy attacks, adversarial attacks against GNNs~\cite{graphatt,nodepoison} have been explored as well as training algorithms to enhance the robustness against such attacks~\cite{robustdef1,robustdef2}.

To mitigate Membership attacks, Memguard~\cite{memguard} and AttriGuard~\cite{attriguard} add carefully crafted noise to the final output prediction to misclassify the shadow model attacks.
Adversarial regularization using minimax optimization regularizes the model to mitigate inference attacks~\cite{advreg}.
Regularization through ensemble training, dropout and L2-regularization have been studied~\cite{ndss19salem}.
Differential Privacy mitigates such privacy attacks with theoretical guarantees by adding noise to gradients but faces an unbalanced privacy accuracy trade-off~\cite{diffpriv}.
Such Differential Privacy frameworks have also been explored in the context of graph and text embeddings~\cite{dptext,dpne} but their efficacy on lowering privacy risks from the proposed attacks is yet to be explored.

\section{Conclusions}\label{conclusions}


The privacy risks of graph embedding algorithms trained on sensitive graph data is not explored and fully-understood.
To this extent, this work provides the first comprehensive privacy risk analysis of publicly released graph embeddings on three major classes of privacy attacks: node membership inference, graph reconstruction and attribute inference attacks.
We propose node membership inference where the adversary aims to infer whether a given user's node was used in the training graph dataset or not.
Next, we show that publicly released embeddings can be inverted to obtain the input graph data enabling an adversary to perform graph reconstruction attack on the sensitive graph data.
This further enables the adversary to perform link inference attack where adversary with a high accuracy can identify whether a link exists between two nodes in the network.
Finally, we show that an adversary can infer sensitive hidden attributes of users such as gender and location from the graph embeddings.
In this work, we successfully perform the above attacks under practical adversary assumptions and threat models to indicate significant privacy leakage from graph embeddings.
This works quantifies privacy risks in graph embeddings and calls for further research to mitigate these privacy threats.


%{\footnotesize
%\bibliographystyle{IEEEtranS}
%\bibliography{paper.bib}
%}

\bibliographystyle{ACM-Reference-Format}
\bibliography{paper}


\end{document}
\endinput
